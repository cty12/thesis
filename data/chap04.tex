\chapter{多虚拟化集群管理系统的设计}
\label{cha:multi-hypervisor-management-system}

通过努力使得 Tsinghua NOVA 系统通过 libvirt 支持了 LXC 虚拟容器,从而从单一的 QEMU-KVM
虚拟集群管理,转变为支持多种虚拟化方案的多虚拟化集群管理系统。而且,管理通过不同的虚拟化技术
运行的虚拟机的用户接口部分是完全相同的,也就是说,用户可以通过一个统一的控制面板 (control
panel) 和相同的操作方式管理由 QEMU-KVM 或是虚拟容器运行的虚拟机。

\section{容器的自动部署}
\label{sec:auto-deployment}

所谓自动部署,无论是对何种虚拟化技术,都是指的从展开客户机需要的操作系统磁盘镜像、到部署客户机的
用户态应用程序、再到连接到管理程序的虚拟机驱动并实现用户对客户机的管理的步骤。在 Tsinghua NOVA
系统中,下层的虚拟化平台对用户来讲是完全透明的。

\subsection{存储方案的比较}
\label{subsec:comparison-network-storage}

既然是一个虚拟化管理平台,那么就需要对虚拟机磁盘镜像或者根文件系统 (rootfs) 进行统一的管理。
在 Tsinghua NOVA 系统中,是由一个主节点 (master node) 提供 web 服务和虚拟机调度服务,
在多个从节点 (slave node) 上部署虚拟集群,所以这个镜像存储服务自然也是将主节点作为服务器、
从节点作为客户端的。

\subsubsection{存储的内容}

首先看一下 QEMU-KVM 的镜像文件和 LXC 的根文件系统 (rootfs) 都具有什么特点。

\subsubsubsection{QEMU 磁盘镜像}

我们在 ~\ref{subsubsec:hardware-virt} 小节提到,QEMU-KVM 模拟设备,比如磁盘设备,
使用的是 QEMU ,因此 QEMU-KVM 支持的磁盘镜像格式就是 QEMU 支持的磁盘镜像格式,主要
包括以下几种:

\begin{itemize}
    \item \textbf{raw:} 简单的二进制镜像,在有的文件系统,比如 xfs 和 ext4 上支持稀疏
    存储的特性,也就是分配的空间会以 metadata 的形式被记录,从而实际占用的空间小于或者等于
    分配的空间;
    \item \textbf{cow:} 简单的写时复制 (copy-on-write) 镜像,这种格式不支持 Windows
    虚拟机;
    \item \textbf{qcow:} QEMU 写时复制 (copy-on-write) 格式,已经被 \textbf{qcow2}
    格式取代了,QEMU 仍然兼容这种格式;
    \item \textbf{qcow2:} 支持快照 (snapshot) 、稀疏存储(即使在不支持上文提到的稀疏
    存储特性的文件系统上也可以实现,不受制于文件系统)、加密、压缩等高级特性的 QEMU 镜像
    格式。这种格式有很好的特性和功能表现,所以是被 Tsinghua NOVA 系统采用用来运行 KVM
    客户虚拟机操作系统的格式;
    \item \textbf{vmdk:} VMWare 的磁盘镜像格式;
    \item \textbf{vdi:} VirtualBox 的磁盘镜像格式;
\end{itemize}

QEMU 对镜像文件的格式支持非常丰富,还有几种不常用的,因为篇幅所限不能一一介绍。Tsinghua NOVA
使用 qcow2 格式,它能通过 QEMU 提供的用户态工具 qemu-img 支持虚拟机增量镜像,从而减少
网络文件系统对镜像存储的负担。因为只要存储一个操作系统的基准镜像,然后每个客户虚拟机创建
自己的增量镜像就可以了。下面简单介绍一下 qemu-img 的常见命令。

创建一个大小为 20GB 、格式为 qcow2 的镜像文件:

\begin{lstlisting}[language=bash]
qemu-img create hello.qcow2 -f qcow2 20G
\end{lstlisting}

查看镜像的详细信息:

\begin{lstlisting}
qemu-img info hello.qcow2
\end{lstlisting}

对镜像进行格式转换,例如把 raw 格式的镜像转为 qcow2 格式的镜像(从而利用 qcow2 格式更好的
特性):

\begin{lstlisting}[language=bash]
qemu-img convert -p -f raw -O qcow2 hello hello2.qcow2
\end{lstlisting}

加上 -p 参数就可以显示进度。

对镜像进行压缩和加密,不过只有 qcow2 格式的镜像才支持这个功能,例如对刚才创建的镜像:

\begin{lstlisting}[language=bash]
qemu-img convert -c -f qcow2 -O qcow2 hello.qcow2 hello3.qcow2
qemu-img convert -f qcow2 -O qcow2 hello.qcow2 hello4.qcow2 -o encryption
\end{lstlisting}

对镜像进行快照,这也是 qcow2 格式的一个特性。例如创建一个 hello.qcow2 的快照:

\begin{lstlisting}[language=bash]
qemu-img snapshot hello.qcow2 -c snapshot00
\end{lstlisting}

创建差量镜像。之所以要使用镜像差量技术,主要是有两个好处:

\begin{enumerate}
    \item \textbf{节约时间:} 通过一个命令可以立即从基镜像生成虚拟机用到的镜像,非常快;
    \item \textbf{节约空间:} 镜像使用写时复制 (copy-on-write) 的模式,读取的时候读取
    自己的或者是基镜像的镜像文件,写入的时候才写入自己的镜像文件,也就是差量镜像,这样
    多个客户虚拟机用一个基镜像,把它们共同的点统一储存,而且只存一份,可以节约空间。
    尤其是对于 Tsinghua NOVA 这种使用网络文件系统的平台,非常重要。
\end{enumerate}

创建的方法:

\begin{lstlisting}[language=bash]
qemu-img create -f qcow2 -b hello.qcow2 hello-bak
\end{lstlisting}

使用 qemu-img 工具还可以很方便地将差量镜像整合回完整的镜像。

\subsubsubsection{LXC 的根文件系统}
\label{subsubsubsec:rootfs-lxc}

LXC 的根文件系统完全是在宿主的文件系统上创建的一个普通的目录,包含客户机的完整的根文件系统,
默认在 /var/lib/lxc 下。由于 namespace 技术将不同容器使用的磁盘资源做了隔离,所以不同
容器可以挂载各自的根文件系统 (rootfs) 并读写之,不会产生干扰。在宿主机上也可以直接对
这些目录进行修改,例如创建文件。

这样的好处是 LXC 的根文件系统 (rootfs) 具有可移动性 (portability) ,比如可以将一个虚拟
容器的文件系统直接拷贝到新的宿主上直接运行,可以保留原先的所有资料不至于丢失。而且在宿主上
可以直接访问客户机的根文件系统,就像一个普通目录一样,这也是一个非常方便的特性。

LXC 的用户态工具提供了一些处理根文件系统 (rootfs) 的手段,但是较为简单和直接,举例如下:

在 /var/lib/lxc/test-fedora-1/rootfs 里创建一个基于 Fedora 22 的根文件系统,
因为需要联网下载所需文件,所以速度主要取决于网速,当然宿主的磁盘 I/O 也是很重要的:

\begin{lstlisting}
lxc-create -n test-fedora-1 -t fedora
\end{lstlisting}

对根文件系统进行快照。所谓的快照就是把根文件系统 (rootfs) 复制一份保留下来:

\begin{lstlisting}
lxc-snapshot -n test-fedora-1
\end{lstlisting}

还有诸如 lxc-clone 等几个功能,总而言之是非常简单的,不仅相比 QEMU-KVM 的镜像组织
\footnote{qcow2 的镜像差量功能相当于自带了一个简单的去冗余}来说,而且相比 docker
完整的镜像定义、组织和类似 git 的版本管理也逊色很多。但是 LXC 的好处是它的基本功能
都比较简单,而且相应的 libvirt 驱动处于一个基本可用的状态,所以 Tsinghua NOVA
系统还是选取 LXC 而不是 docker 作为 Linux 虚拟容器的解决方案。当然,docker 的
“一揽子”用户态工具是非常非常方便的,类似的基于 web 的图形管理控制面板也有不少。
\footnote{比如:https://shipyard-project.com/}

\subsubsection{几种网络存储方案的选型}
\label{subsubsec:network-storage}

明确了要存储什么样的镜像资料之后,就要开始对网络存储方案进行选型。网络存储主要是为了
在创建虚拟机的时候能及时调用相应的镜像文件,从而创建并运行新的客户机。主要考察了
SSHFS / SMB / NFS 几种网络文件存储协议和文件系统。

SSHFS 的优点是它基于 FUSE 。FUSE 是一种用户态文件系统框架,可以很方便地创造用户态
文件系统。文件系统跑在用户态的好处是即使它崩掉了也不会影响系统内核的运行,比如我们
需要一个可以将云盘挂载到本地的文件系统,那么最好让它跑在用户态,因为这个功能对于操作系统
内核来讲通常是无关紧要的,即使崩溃了也没关系。基于 FUSE 文件系统的一个例子是笔者
写的 sdbfs \footnote{Simple database file system:
https://github.com/cty12/sdbfs} ,它能将文件系统的读写自动转化为 sqlite 的数据库操作。

SSHFS 的缺点也很明显,就是它是基于 SSH (secure shell) 的,创建文件只能创建使用该账户
连接的那个用户所有的,而且在一个连接内不能方便地切换用户。对于我们的需求,这个特点显得
不够灵活。而且 Tsinghua NOVA 系统因为历史遗留原因需要 root 权限运行。

SMB (Server Message Block) 是微软 Windows 系统默认使用的一种文件共享协议,可以方便
地启动“共享文件夹”\footnote{“目录” (directory) 在微软的操作系统下称为“文件夹” (folder)}
功能。SMB 服务有开源的实现叫做 samba ,被主流的 GNU/Linux 发行版包括在软件源内,但是
微软的闭源(私有)实现和 samba 的开源实现有一定的差距,而且这个差距会反映在两者的性能上。
而且 samba 的实现在小文件的读写上,和 NFS 具有一定的差距\footnote{https://
ferhatakgun.com/network-share-performance-differences-between-nfs-smb/}。

NFS (Network File System) 是一种完全开放的分布式文件系统标准,支持大多数平台,而且性能
也相对比较好。在 Tsinghua NOVA 系统中,我们使用它完成客户机操作系统镜像的共享和存储。

\subsection{虚拟容器的自动部署和启动}

在之前的章节~\ref{subsubsubsec:rootfs-lxc}和~\ref{subsubsec:network-storage},
我们介绍了 Linux 虚拟容器的文件系统存储结构以及 Tsinghua NOVA 管理系统使用的网络存储
方案。在这一节,介绍 Tsinghua NOVA 系统是如何利用前面介绍的相关工具的特点实现 Linux
虚拟容器客户机操作系统的自动部署和启动的。

\subsubsection{根文件系统的展开}

Tsinghua NOVA 系统的 Linux 虚拟容器支持有一个非常重要的特点,就是支持容器用户态负载的
迁移 (migration) ——而且是动态迁移 (live migration)。这个特性会在后面的章节重点讲解,
在本小节,先介绍实现动态迁移技术的文件系统基础。

通过对 LXC 的介绍,认识到如果在两台宿主机之间迁移一个虚拟容器,那么一个朴素 (trivial)
的想法就是通过网络把根文件系统 (rootfs) 从源宿主机 (source host) 传输到目标宿主机
(destination host) 上。这个传输最朴素的话可以使用 scp 实现,但是这样的话,每次虚拟容器
迁移都要占用很高的带宽。

Tsinghua NOVA 系统在运行过程中实现了一个动态调度特性\footnote{这个功能目前还是初步且
实验性的,所以没有成为本论文的一部分},也就是所谓的 LoadBalancingDaemon 。负载均衡
(load balancing) 是一个非常大的概念,流行的框架比如 HAProxy
\footnote{http://www.haproxy.org/} 是在 TCP/HTTP 请求 (request) 的层面上做均衡,
而对于一个虚拟化管理系统,也应该考虑在客户虚拟机负载情况的角度做负载均衡,也就是虚拟机的
调度 (scheduling) 。支持这个技术的虚拟化管理系统面临的一个挑战就是如何减小虚拟机迁移
的开销 (overhead) 。因为如果虚拟机迁移的开销非常大的话,那么就很容易抵消掉做虚拟机调度
带来的性能提升,因为这种提升本质上是靠进行了虚拟机迁移带来的。当然,理想情况是虚拟机迁移
根本不需要开销,也就是一个虚拟机“一下子”从物理集群中的一台宿主机跑到了另一台宿主机上接着
运行。但是这个是不可能的,因为保存现场,也就是虚拟机(或者是虚拟容器,对于虚拟容器的情况
我们下面会看到)中负载的进程的运行状态以及恢复现场肯定是需要时间的,而且整个系统的 RPC
通讯也需要时间,因为需要告诉目标宿主机原来的宿主上的虚拟机已经被调度器 (scheduler) 杀死了。

综上,在现实中虚拟容器迁移的开销越小越好。针对文件系统的迁移自然也是如此,那么有没有办法将
文件系统的迁移开销减少到 0 呢?是有的。这也就是之所以 Tsinghua NOVA 要把 NFS 作为依赖的
原因。

在 Tsinghua NOVA 系统中,既使用 NFS 来保存使用的操作系统根文件系统,也用 NFS 来保存
每一个虚拟容器的根文件系统。这个根文件系统在虚拟容器创建之时完全是操作系统根文件系统压缩包
的一个复制。解包出这个文件系统肯定是需要开销的,但是我们将在后面的章节的实验中说明这个开销
在我们的测试环境中表现得完全是可以接受的。

% TODO 这里应该有一个创建虚拟容器的 RPC 流程图

虽然 Tsinghua NOVA 系统会通过 RPC 通知相应的从节点新的虚拟容器应该被创建,但是解压缩
根文件系统这个步骤是在主节点操作的。这是基于两个事实:一是我们的系统里运行 web 界面服务的
HTTP server 是跑在主节点上的
