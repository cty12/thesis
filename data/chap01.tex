\chapter{引言}
\label{cha:intro}

\section{选题背景}

\subsection{云计算的历程}

云计算的概念诞生的远远比它的名字更早。它的概念可以追溯到早期的
分时系统\footnote{time-sharing systems}。

早期的计算机系统价格昂贵,而且体积庞大,使得个人用户很难拥有独立的个人计算机 (PC) 。
这就导致了多个用户可以同时使用同一个计算资源——例如一台电子计算机——的分时系统
~\cite{timesharing}的产生。所谓分时系统,就是有一个主要的
计算系统 (例如大型机,mainframe computer) ,用户通过终端机 (terminal) 连接到这个
系统上使用计算资源。用户一般而言不需要考虑计算机操作系统乃至硬件的具体细节——因为自然
有管理员管理这些细节,只需要像“租客”一样使用计算资源就可以了。现代的操作系统,
例如 GNU/Linux 或者 Microsoft Windows ,都支持多用户模式。

\begin{figure}[h]
    \centering
    \includegraphics[width=0.7\textwidth]{unix_time_sharing}
    \caption{威斯康辛大学麦迪逊分校的学生使用终端机,1978 年}
\end{figure}

从 2000 年左右开始,云计算的概念正式形成了。2006 年诞生的 Amazon EC2 \footnote{EC2
  是 Elastic Compute Cloud 的缩写,因有两个连着的 C 所以叫做 EC2 ,不是
  “第二个版本”的意思。}成为了迄今最成功的云计算平台之一。2008 年诞生的 OpenNebula
是另一个云计算平台,顾名思义它是一个自由软件\footnote{Free as in freedom}
,使用 Apache License 发布。

而现在更加流行的自由的云计算平台是 OpenStack 。OpenStack 最初是由
 NASA 和 RackSpace 共同发起的,从 2012 年开始由叫做 OpenStack Foundation 的非营利
组织进行开发和维护。它和 Amazon EC2 的最大不同就是它是完全自由和开源的。不但可以利用
这个软件搭建自己的云计算环境,而且可以对相关的代码展开研究。OpenStack 中的主要组成部分
和 Amazon AWS 中的部分具有一定的对应关系,比如 AWS 的核心也就是 EC2 对应了 OpenStack
 里的 Nova ,负责虚拟机的调度和管理;AWS 中的简单存储模块 S3 在 OpenStack 中
有 Swift 与之对应~\cite{openstack}。

\subsection{云计算平台的服务类型}

\begin{figure}[h]
    \centering
    \includegraphics[width=0.7\textwidth]{iaas-paas-saas}
    \caption{三种服务类型,灰色部分是由平台提供商管理的部分}
\end{figure}

云计算提倡的概念是“所有都是服务”
 (everything as a service) ~\cite{cloud-and-openstack},也就是用户不直接接触
物理上的计算机,但是却能通过网络获得相应的计算服务。云计算的“云”不在本地,这就好比大家
每天日常生活都要使用电,但极少有人在家里自己搭建一台柴油发电机,而是在发电厂集中发电,
然后通过输电线路把电力输送到千家万户供人们使用。数据中心就好像这个发电厂,计算机网络
就好像电线,终端用户就像使用电力一样使用云计算产生的计算资源。

根据服务类型不同,云计算平台主要可以分为以下三种——基础设施
服务 (IaaS)、平台服务 (PaaS) 还有软件服务 (SaaS) ~\cite{types-of-cloud}。

\subsubsection{基础设施服务}

顾名思义,这种服务类型就是只提供必要的设施,而不提供上层应用。这些设施包括由虚拟机管理程序
 (hypervisor) 支持的大量的虚拟机集群、操作系统磁盘镜像存储池(这样用户在创建虚拟机实例的
时候就不用联网从镜像源下载镜像,直接从镜像存储服务调取相应的镜像即可)、文件存储服务、
防火墙、负载均衡器等等。

当然还有虚拟局域网 (VLAN)。例如某大学计算机系有网络和操作系统两个实验室,它们共享一套虚拟
机集群。网络实验室和操作系统实验室希望各占用一个子网。如果是使用物理上的局域网,那么应该有
两个交换机,各接入路由器的两个物理端口上。而使用了虚拟局域网,就可以用两个逻辑端口代替。
使用虚拟局域网,可以缩小广播域,从而提高集群的安全性。

\begin{figure}[h]
    \centering
    \includegraphics[width=\textwidth]{openstack-conceptual-diagram}
    \caption{OpenStack 的概念架构图(来自 OpenStack 官方电子手册)}
\end{figure}

上文提到的 Amazon EC2 就是一个典型的 IaaS 平台。类似的,OpenStack 也是一个 IaaS 平台,
截至到 2016 年,一共包含以下几个核心组成部分~\cite{openstack}:

\begin{enumerate}
    \item \textbf{Nova:} Nova 是 OpenStack 的计算部分,主要负责虚拟机管理。它的 API
    可以和 Amazon EC2 的相兼容,RackSpace 和惠普~\cite{hpe-openstack}的商业计算服务
    就是建立在 Nova 上的。Nova 主要包含以下几个组件\footnote{实际上还有 nova-cert
     和 nova-objectstore ,但不是主要的}:
    \begin{figure}[h]
        \centering
        \includegraphics[width=0.7\textwidth]{openstack-nova-arch}
        \caption{OpenStack Nova 的架构图}
    \end{figure}
    \begin{enumerate}
        \item \textbf{nova-compute:} 这是 Nova 用来管理虚拟机的模块。它安装在物理集群
        中的每台机器上,主要负责接收用户对虚拟机的操作,例如创建删除等。这些操作依赖底层的
        虚拟机管理程序 (hypervisor) 提供的 API ,如 XenAPI 、VMWareAPI
         或者 libvirt 。
        \item \textbf{nova-scheduler:} 是一个调度器,负责协调例如在哪台物理机上创建
        新的虚拟机这样的调度工作。可以灵活的通过配置文件配置。
        \item \textbf{nova-conductor:} 负责安全的组件。nova-compute 进行的数据库
        操作,需要由 nova-conductor 代为转交给 nova-db 进行处理。它设计的目的是为了
        防止 nova-compute 直接访问数据库,所以不能和 nova-compute 一样直接部署在
        物理计算节点上,不然就失去了保障安全的效果了。~\cite{conductor}
        \item \textbf{nova-db:} 数据库,记录租户信息、虚拟机状态、物理机状态等。实际上
        因为 nova 是使用 Python 实现的,所以数据库接口部分使用的是 SQLAlchemy
         \footnote{\url{http://www.sqlalchemy.org/}},可以使用任何 SQLAlchemy
         兼容的数据库。
        \item \textbf{nova-console:} 顾名思义,Nova 的控制台服务。
    \end{enumerate}
    \item \textbf{Neutron:} OpenStack 的虚拟网络服务叫做 Neutron ,它的前身是
     nova-network 。它的理念是“网络作为服务”\footnote{Network as a service}。
    它允许用户创建虚拟网络并连接网络设备的接口。Neutron 有一个主要的服务进程
     neutron-server 运行在网络控制节点上,它接受用户发来的 HTTP 请求,发送给遍布
    物理集群的机器上运行的 agent 进行处理。为了更容易扩展,Neutron 提供 plugin 机制,
    每个 plugin 提供一组特定的 API ,通过 RPC 调用 agent 完成操作。
    \begin{figure}[h]
        \centering
        \includegraphics[height=0.5\textheight]{openstack-swift-arch}
        \caption{OpenStack Swift 的架构图}
    \end{figure}
    \item \textbf{Swift:} Swift 项目是 OpenStack 提供的对象存储 (Object Storage)
     服务\footnote{对于程序员来说,这个概念容易和“面向对象”中的对象相混淆,前者是指在
    云端存储的一个文件,后者是对具有一定行为特征的一类概念的抽象}。一个 IaaS 服务框架,
    为什么要提供存储服务呢?物理世界中,一般使用存储区域网络
     (SAN) 或者网络附属存储 (NAS) 来保证数据的集中管理。在虚拟化的世界中,虚拟机数据的
    集中存储更是至关重要,因为虚拟机的临时存储好比无源之水、无本之木,随着虚拟机的删除,
    它存储的数据也就消失了。对象存储和快存储就是为了解决这个问题,好把数据统一管理起来。
    OpenStack 里的对象存储就是 Swift ,Swift 又分为访问层和存储层两部分,分别负责
    RESTful 请求的处理、权限控制和实际对象数据的存储。Swift 主要应用在不容易发生变化的
    数据文件上,例如磁盘镜像文件和备份文件,当然还有大的媒体文件。
    \begin{figure}[h]
        \centering
        \includegraphics[width=0.7\textwidth]{openstack-cinder-arch}
        \caption{OpenStack Cinder 的架构图}
    \end{figure}
    \item \textbf{Cinder:} Cinder 项目是 OpenStack 提供的块存储 (Block Storage)
     机制,类似于 Amazon EBS (Elastic Block Storage) 。Cinder 提供了逻辑
    存储卷 (volume)的抽象,举例来说,如果数据库服务在一个由 Nova 提供的虚拟机里运行,
    那么一旦虚拟机出错,存储在 Cinder 提供的逻辑存储卷里的数据库还能通过附加 (attach)
     到新的虚拟机上来很快恢复服务,不至于出现“鸡飞蛋打”的情况。从这个意义上,一个 volume
     好比一块移动硬盘,可以在虚拟机之间插拔 (attach 和 detach) 。~\cite{cinder}
    \item \textbf{Glance:} Glance 是 OpenStack 存储三驾马车的最后一架,它是一个镜像
    管理项目,但是本身不负责存储,存储需要依赖 Swift 等项目来实现。它的 API 主要提供了
    镜像 (image) 的导入导出、镜像元数据 (metadata) 的管理等。
    镜像主要有 queued 、saving 、 active 、 killed 、
     deleted 、 pending\_delete 这六个状态。
    \item \textbf{Keystone:} OpenStack 的身份管理系统。
\end{enumerate}

OpenStack 作为 IaaS 开源界的新秀,现在还在不断地发展、完善。

\subsubsection{平台服务}

\subsubsection{软件服务}

\subsection{虚拟化和云计算}

\section{研究价值和主要贡献}

\section{论文结构}
