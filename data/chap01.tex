\chapter{引言}
\label{cha:intro}

\section{选题背景}

\subsection{今天的云计算}

云计算的概念诞生的远远比它的名字更早。它的概念可以追溯到早期的
分时系统\footnote{time-sharing systems}。早期的计算机系统价格昂贵,而且体积庞大,
使得个人用户很难拥有独立的个人计算机 (PC) 。这就导致了多个用户可以同时使用同一个计算
资源——例如一台电子计算机——的分时系统~\cite{timesharing}的产生。所谓分时系统,就是
有一个主要的计算系统 (mainframe computer) ,用户通过终端机 (terminal) 连接到这个
系统上使用计算资源。用户一般而言不需要考虑计算机操作系统乃至硬件的具体细节——因为自然
有管理员管理这些细节,只需要像“租客”一样使用计算资源就可以了。现代的操作系统,
例如 GNU/Linux 或者 Microsoft Windows ,都支持多用户模式。

从 2000 年左右开始,云计算的概念正式形成了。2006 年诞生的 Amazon EC2 \footnote{EC2
  就是 Elastic Compute Cloud 的缩写,因为有两个连着的 C 所以叫做 EC2 ,要注意不是
  “第二个版本”的意思。}成为了迄今最成功的云计算平台之一。2008 年诞生的 OpenNebula
是另一个云计算平台,顾名思义它是一个自由软件\footnote{Free as in freedom}
,使用 Apache License 发布。

相比 OpenNebula ,现在更加流行的自由的云计算平台是 OpenStack 。

\subsection{云计算的服务类型—— SaaS, PaaS 和 IaaS}

\subsection{虚拟化和云计算}

\section{研究价值和主要贡献}

\section{论文结构}
