\chapter{后续工作及结论}

\section{可能的改进}

\subsection{对于 Tsinghua NOVA}

Tsinghua NOVA 系统现在还有很多不足之处。迫切需要改进的地方包括以下几点:

\begin{enumerate}
    \item 将 Java Jar 依赖管理迁移到 Maven 。目前的管理方法是使用 git ,但是 git
    本身的特性决定了它不适合管理二进制文件;
    \item 更好的虚拟机管理程序抽象层。之前提到,OpenStack 有很好的 hypervisor 抽象,
    这正是 Tsinghua NOVA 系统目前缺乏的。在很多地方还是使用原始的 if-else 条件判断来
    针对不同的虚拟化引擎做不同的操作,这样就给增加新的虚拟化支持带来了困难;
    \item 镜像的去冗余存储。如果说对于 KVM 来讲这个需求还不甚迫切,因为 qcow2 格式支持
    差量镜像,那么对于 LXC 的根文件系统 (rootfs) 来说,去冗余存储就是非常关键的。如果
    加入了去冗余机制,那么每个容器对应的目录占用的磁盘空间将大大减小;
    \item 减小迁移的开销。现在 LXC 迁移的开销比较大,还不够实用;
    \item 更有效的客户机调度机制;
    \item 更友好的网页端;
    \item 更多的虚拟机管理程序的支持。
\end{enumerate}

\subsection{对于 Omegabench 基准测试套件}

目前,Omegabench 是一个结合了多种常见的负载的一个自动化性能评估与测试工具,但是它的泛用性
不够好,增加新的测试比较困难。这也是因为对于下层的测试用的服务和客户端程序做的抽象不够彻底。
比较好的改进方向是作为一个 libbench ,即多种基准测试工具都可以使用统一的上层 API 进行
调用,而 API 上面又提供给终端用户现成的命令行工具。

因为目前开源的高质量基准测试程序还不多,所以这个项目是很有前景的。

\section{总结}

在本文中,提出了一种结合了新兴的虚拟容器技术的多虚拟化管理平台。这个平台在不同种类的、完全
不同原理的虚拟化技术之上,构建了一个统一的用户接口,使得用户可以通过相似的方式,管理由
多种虚拟化技术产生的客户机的集群。在这个平台上,进一步实现了虚拟容器的实时迁移,尽管开销
还比较大,但是能较为鲁棒地将负载的状态转移到另外一台宿主物理机上。

另外,介绍了一个新的基准测试套件。它包含几种虚拟集群中常见的负载服务,尽可能地模仿真实的
应用场景。通过它得到的实验结果,对于虚拟化技术选型,还有容器虚拟技术的改进,都有一定的
指导意义。
