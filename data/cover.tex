\thusetup{
  %******************************
  % 注意:
  %   1. 配置里面不要出现空行
  %   2. 不需要的配置信息可以删除
  %******************************
  %
  %=====
  % 秘级
  %=====
  secretlevel={秘密},
  secretyear={10},
  %
  %=========
  % 中文信息
  %=========
  ctitle={虚拟容器管理系统设计与实现},
  cdegree={工学学士},
  cdepartment={计算机科学与技术系},
  cmajor={计算机科学与技术},
  cauthor={陈天昱},
  csupervisor={姜进磊副教授},
  % 日期自动使用当前时间,若需指定按如下方式修改:
  % cdate={超新星纪元},
  %
  % 博士后专有部分
  cfirstdiscipline={计算机科学与技术},
  cseconddiscipline={系统结构},
  postdoctordate={2009年7月——2011年7月},
  id={编号}, % 可以留空: id={},
  udc={UDC}, % 可以留空
  catalognumber={分类号}, % 可以留空
  %
  %=========
  % 英文信息
  %=========
  etitle={The Design and Implementation of a Linux Container Management System},
  % 这块比较复杂,需要分情况讨论:
  % 1. 学术型硕士
  %    edegree:必须为Master of Arts或Master of Science(注意大小写)
  %             “哲学、文学、历史学、法学、教育学、艺术学门类,公共管理学科
  %              填写Master of Arts,其它填写Master of Science”
  %    emajor:“获得一级学科授权的学科填写一级学科名称,其它填写二级学科名称”
  % 2. 专业型硕士
  %    edegree:“填写专业学位英文名称全称”
  %    emajor:“工程硕士填写工程领域,其它专业学位不填写此项”
  % 3. 学术型博士
  %    edegree:Doctor of Philosophy(注意大小写)
  %    emajor:“获得一级学科授权的学科填写一级学科名称,其它填写二级学科名称”
  % 4. 专业型博士
  %    edegree:“填写专业学位英文名称全称”
  %    emajor:不填写此项
  edegree={Bachelor of Engineering},
  emajor={Computer Science and Technology},
  eauthor={CHEN Tianyu},
  esupervisor={Associate Professor JIANG Jinlei},
  % 日期自动生成,若需指定按如下方式修改:
  % edate={December, 2005}
  %
  % 关键词用“英文逗号”分割
  ckeywords={虚拟化, 虚拟容器, 分布式系统, 性能测试, 应用自动部署},
  ekeywords={Virtualization, Linux Container, Distributed Systems, Benchmark, Application Automatic Deployment}
}

% 定义中英文摘要和关键字
\begin{cabstract}
以 OpenStack 为例的现有虚拟化管理平台软件栈对传统的虚拟化技术,尤其是开源虚拟化技术
QEMU-KVM 支持较好,而对新兴的虚拟容器技术支持不佳,甚至基本上没有可用的支持。以 LXC
虚拟容器的用户态工具为例,只有简单的命令行工具,大规模管理较为困难。

面对这种情况,本文将 LXC 虚拟容器的支持加入了虚拟集群管理系统 Tsinghua NOVA ,从而
使之具有了管理传统虚拟化和容器虚拟化混合集群的能力,用户可以通过简单的网页交互自由选择
运行客户机的下层虚拟化平台。本系统做到了虚拟化平台对于用户的透明。

本文的另一个贡献是设计实现了一套虚拟集群基准测试系统。该系统通过模拟运行现实中虚拟集群
常见的负载,对不同的虚拟化平台的性能优劣进行了评估,得出了虚拟化选型的参考。
\end{cabstract}

% 如果习惯关键字跟在摘要文字后面,可以用直接命令来设置,如下:
% \ckeywords{\TeX, \LaTeX, CJK, 模板, 论文}

\begin{eabstract}
Existing virtual cluster management systems such as OpenStack adopt traditional
hypervisor QEMU-KVM as the major virtualization driver but provide relatively
poor support for the emerging \"lightervisor\" technologies such as LXC. Meanwhile
the native user-land tools for LXC, which support nothing but shell commands,
are simple and user-unfriendly for managing a large cluster of guests.

The LXC driver is added to Tsinghua NOVA so as to resolve the dillemma between
container technology and large scale cluster management. Tsinghua NOVA is
capable of hosting a hybrid cluster made of QEMU-KVM enabled guests and
container enabled guests. The whole system is transparent to end-users.
System administrators select the virtualization technology by clicking on
a web page, which is very intuitive.

A virtual cluster benchmark suite is implemented, which is another contribution
of this project. The performance of multiple virtualization drivers is
evaluated based on simulated real-world payload. Suggestions on choosing
appropriate hypervisor technology are made according to the data collected
in the experiments.
\end{eabstract}

% \ekeywords{\TeX, \LaTeX, CJK, template, thesis}
