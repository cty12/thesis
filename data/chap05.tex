\chapter{虚拟集群测试套件的设计}

在本章中,介绍一种基于真实负载类型的虚拟集群测试套件 (benchmark suite) 。这个套件是
Tsinghua NOVA 项目的一部分,用于对不同的虚拟化技术进行性能和实用性评估。

\section{负载类型的选择}

常见的基准测试软件可以按照负载类型分成两类。一种是运行理想化的负载,例如计算圆周率的
SuperPI 。在实际应用中,恐怕没有用户会每天使用计算机无穷无尽地计算圆周率,所以这种
负载只能说是对真实世界的计算问题的一种简单的抽象。另一种是运行现实中真实存在的综合性
的计算需求,比如 SPEC CPU 基准测试,包括了许多科学计算任务——流体力学计算、声音识别
等等,又比如 GeekBench ,套件中包含了图象压缩、数据加密、哈希计算等多种个人用户常用的
计算功能的性能评估。

本测试套件 Omegabench 主要是面向大型服务器集群,尤其是虚拟化集群。在本测试套件中,
主要包含了以下几种类型的负载:

\begin{enumerate}
    \item \textbf{web 服务器: } 主要包括传统的 HTTP 服务器 Apache httpd 和轻量级
    的 HTTP 服务 mongoose 两个测试项目。基准测试使用的客户端是 Apache 自带的测试程序,
    适用于几乎所有的 HTTP 服务。主要测试指标是服务器的相应速度和吞吐量;
    \item \textbf{存储和数据库服务: } 这个类别包括三个功能互不相同的服务。一是 memcached
    ,一个通用的分布式数据内存缓存 (memory caching) 服务,二是 redis ,一个内存数据存储
    服务,既可以作为缓存 (caching) 也可以作为一个内存数据库,三是 mysql ,一个著名的
    数据库服务;
    \item \textbf{压缩和加密: } 这个类别目前只包含一个测试项目,也就是 xz 的解压缩以及
    tar 解包的综合测试。这个测试既考验 CPU 性能也考验磁盘 I/O 性能。
\end{enumerate}

\section{部署本测试套件}

\subsection{目录结构}

\subsection{安装依赖和编译测试服务}

\subsection{运行测试程序}

\section{为套件添加新的测试用例}
