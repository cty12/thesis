\chapter{系统的测试与评价}

\section{不同虚拟化方案对负载性能影响的比较}

在本节,首先介绍 Tsinghua NOVA 项目中的基准测试套件 Omegabench ,然后通过在不同的
虚拟化平台上运行同样的模仿真实环境的负载,对不同虚拟化方案的效果和实用性作出评估。

\subsection{实验环境}

实验使用的硬件是一台清华同方“超强”系列的塔式服务器,型号是 TP-240 。具体配置如下:

\begin{lstlisting}
Tsinghua-Tongfang TP-240 tower server
    |-- Dual Intel Xeon E5606 @ 2.13 GHz
    |-- 32 GB DDR3 ECC memory
    |-- WDC WD10EALX-229
\end{lstlisting}

宿主机使用的操作系统是 CentOS 7.2 ,也就是目前最新的版本。相应的软件栈也在实验之前更新
到 Red Hat EPEL 软件仓库中的最新版本。

\subsection{负载类型的选择}

常见的基准测试软件可以按照负载类型分成两类。一种是运行理想化的负载,例如计算圆周率的
SuperPI 。在实际应用中,恐怕没有用户会每天使用计算机无穷无尽地计算圆周率,所以这种
负载只能说是对真实世界的计算问题的一种简单的抽象。另一种是运行现实中真实存在的综合性
的计算需求,比如 SPEC CPU 基准测试,包括了许多科学计算任务——流体力学计算、声音识别
等等,又比如 GeekBench ,套件中包含了图象压缩、数据加密、哈希计算等多种个人用户常用的
计算功能的性能评估。

本测试套件 Omegabench 主要是面向大型服务器集群,尤其是虚拟化集群。在本测试套件中,
主要包含了以下几种类型的负载:

\begin{enumerate}
    \item \textbf{web 服务器: } 主要包括传统的 HTTP 服务器 Apache httpd 和轻量级
    的 HTTP 服务 mongoose 两个测试项目。基准测试使用的客户端是 Apache 自带的测试程序,
    适用于几乎所有的 HTTP 服务。主要测试指标是服务器的相应速度和吞吐量;
    \item \textbf{存储和数据库服务: } 这个类别包括三个功能互不相同的服务。一是 memcached
    ,一个通用的分布式数据内存缓存 (memory caching) 服务,二是 redis ,一个内存数据存储
    服务,既可以作为缓存 (caching) 也可以作为一个内存数据库,三是 mysql ,一个著名的
    数据库服务;
    \item \textbf{压缩和加密: } 这个类别目前只包含一个测试项目,也就是 xz 的解压缩以及
    tar 解包的综合测试。这个测试既考验 CPU 性能也考验磁盘 I/O 性能。
\end{enumerate}

\subsection{运行本测试套件}

我们的测试套件的远端仓库 (repository) 托管在 GitHub 上,使用 git 进行版本控制。可以
以源代码的形式获得这个套件。

\subsubsection{目录结构}

工程的根目录包含一个叫 tests/ 的目录,和其它若干个以测试用例的名称命名的目录,比如 mongoose/
和 redis/ 等。tests/ 目录包含了全部测试脚本,和测试使用的配置文件,还有以表格或者图表的形式
收集测试结果的相关代码。而以测试项目命名的目录中一般有一个用于自动下载和编译安装测试用的服务
的脚本,还有一个开启服务器的脚本。通过先后运行这两个脚本可以完成测试用服务的自动部署和配置。
还有一个文件保存编译的时候使用的配置信息,比如服务器软件的版本、所需依赖的版本等。

\subsubsection{安装依赖和编译测试服务}

以 Apache httpd 为例,首先通过保管理器安装相关的依赖:

\begin{lstlisting}[language=bash]
dnf install tar wget # required when using Fedora mininal installation
dnf install gcc make
dnf install pcre-devel
dnf install libxml2-devel
dnf install xz       # for testing xz and tar
\end{lstlisting}

然后需要在相应的 shell 配置文件比如 .bashrc 中定义 \$OMEGA_HOME 这个变量:

\begin{lstlisting}[language]
export OMEGA_HOME="your/preferred/dir/here"
\end{lstlisting}

然后运行叫做 \$OMEGA_HOME/apache-httpd/download-compile-apache.sh 即可自动完成
httpd 以及 PHP 解释器的编译安装,而且会自动生成一个测试用的配置文件 httpd.conf 。使用
apachectl 可以控制 httpd 的运行与停止,通过 -f 参数可以指定配置文件。

\subsubsection{运行测试程序}

仍然以 Apache httpd 举例,通过运行 run-ab-tests.py 可以自动进行基准测试并以列表和
柱状图的形式显示出性能数据。

相应的测试程序配置文件是 ab-tests-config.json ,一个简单的测试本机的例子如下:

\begin{lstlisting}
{
  "ab": "$OMEGA_HOME/apache-httpd/install/bin/ab",
  "num_req": 128,
  "concurrency": 8,
  "port": 7000,
  "ip_addr": [
    "127.0.0.1"
  ]
}
\end{lstlisting}

在这个例子中,测试客户端一共向服务器发送 128 个 HTTP 请求 (requests) ,
并发度 (concurreny level) 是 8 ,连接到服务器的 7000 端口,服务器的 IP 地址是
127.0.0.1 。

\subsection{实验结果}

Mongoose 的测试结果如表~\ref{tab:mongoose-perf}。

\begin{table}[h]
    \centering
    \caption{Mongoose 的性能测试结果}
    \begin{tabular}{||c c c c||}
        \hline
        FIELD & KVM & LXC & BAREMETAL \\
        \hline
        \hline
        Time per request (ms) & 255.608 & 267.923 & 284.447 \\
        \hline
        Requests per sec & 31.3 & 29.86 & 28.12 \\
        \hline
        Transfer rate (KB/s) & 3.39 & 3.24 & 3.05 \\
        \hline
        Time taken (s) & 4.09 & 4.287 & 4.551 \\
        \hline
    \end{tabular}
    \label{tab:mongoose-perf}
\end{table}

\section{调度性能评估}

\subsection{实验环境}

\subsection{迁移开销}
